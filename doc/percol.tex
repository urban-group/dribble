\documentclass{scrartcl}

\usepackage[utf8]{inputenc}
\usepackage[T1]{fontenc}

\usepackage{amsmath}
\usepackage{amssymb}

\usepackage[osf,sc]{mathpazo}
\usepackage[scaled=0.86]{berasans}
\usepackage[scaled=1.03]{inconsolata}

\setlength{\parindent}{0pt}
\setlength{\parskip}{0.5em}

\begin{document}

\section{A take on percolation theory}

The problem of percolation through a regular lattice can be approached
from two directions: 
%
in the limit of \emph{site percolation} only the decoration of the
lattice sites is cnsidered.  If two neighboring sites are occupied by
the percolating species they are assigned to one \emph{cluster} of
sites.  We shall denote $P_{n}^{\,\textup{site}}$ the probability of
finding a cluster of $n$ sites on a given lattice.  The system is
percolating, if it exhibits at least one infinitely extended cluster.
%
Let $P_{\infty}^{\,\textup{site}}$ be the probability that such an
inifinite cluster exists.
%
Given the probability $p$ of a lattice site being occupied by the
percolating species there exists a critical concentration $p =
p_{\textup{c}}^{\,\textup{site}}$, the site percolation threshold, below
which $P_{\infty}$ is exactly equal to zero.

The second limit for the description of percolation considers the
\emph{bonds} between sites.  The same 


\subsection{Site percolation}

For an infinite lattice with a concentration $p$ of the conducting
species, the \emph{percolation probability} $P_{\infty}(p)$, i.e., the
probability that an occupied site is part of a percolating (infinite)
cluster is given by
%
\begin{align}
  P_{\infty}(p) 
  = \begin{cases}
    \hat{B}_{p}\,
    \Bigl( 
      \frac{p}{p_{\textup{c}}} - 1
    \Bigr)^{\beta_{p}}
    & \text{for}\quad p>p_{\textup{c}}
    \\
    0 & \text{else}
    \quad ,
  \end{cases}
  \label{eq:P_infty}
\end{align}
%
where the critical concentration $p_{\textup{c}}$ is the \emph{site
  percolation threshold}.  The prefactor $\hat{B}_{p}$ and the exponent
$\beta_{p}$ both depend on the lattice and the concentration.

\end{document}

%%% Local Variables: 
%%% mode: latex
%%% TeX-master: t
%%% End: 
